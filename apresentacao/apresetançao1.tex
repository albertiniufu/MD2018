\documentclass[]{article}
\usepackage{lmodern}
\usepackage{amssymb,amsmath}
\usepackage{ifxetex,ifluatex}
\usepackage{fixltx2e} % provides \textsubscript
\ifnum 0\ifxetex 1\fi\ifluatex 1\fi=0 % if pdftex
  \usepackage[T1]{fontenc}
  \usepackage[utf8]{inputenc}
\else % if luatex or xelatex
  \ifxetex
    \usepackage{mathspec}
  \else
    \usepackage{fontspec}
  \fi
  \defaultfontfeatures{Ligatures=TeX,Scale=MatchLowercase}
\fi
% use upquote if available, for straight quotes in verbatim environments
\IfFileExists{upquote.sty}{\usepackage{upquote}}{}
% use microtype if available
\IfFileExists{microtype.sty}{%
\usepackage{microtype}
\UseMicrotypeSet[protrusion]{basicmath} % disable protrusion for tt fonts
}{}
\usepackage[margin=1in]{geometry}
\usepackage{hyperref}
\hypersetup{unicode=true,
            pdftitle={projeto1},
            pdfauthor={Bernardo},
            pdfborder={0 0 0},
            breaklinks=true}
\urlstyle{same}  % don't use monospace font for urls
\usepackage{color}
\usepackage{fancyvrb}
\newcommand{\VerbBar}{|}
\newcommand{\VERB}{\Verb[commandchars=\\\{\}]}
\DefineVerbatimEnvironment{Highlighting}{Verbatim}{commandchars=\\\{\}}
% Add ',fontsize=\small' for more characters per line
\usepackage{framed}
\definecolor{shadecolor}{RGB}{248,248,248}
\newenvironment{Shaded}{\begin{snugshade}}{\end{snugshade}}
\newcommand{\KeywordTok}[1]{\textcolor[rgb]{0.13,0.29,0.53}{\textbf{#1}}}
\newcommand{\DataTypeTok}[1]{\textcolor[rgb]{0.13,0.29,0.53}{#1}}
\newcommand{\DecValTok}[1]{\textcolor[rgb]{0.00,0.00,0.81}{#1}}
\newcommand{\BaseNTok}[1]{\textcolor[rgb]{0.00,0.00,0.81}{#1}}
\newcommand{\FloatTok}[1]{\textcolor[rgb]{0.00,0.00,0.81}{#1}}
\newcommand{\ConstantTok}[1]{\textcolor[rgb]{0.00,0.00,0.00}{#1}}
\newcommand{\CharTok}[1]{\textcolor[rgb]{0.31,0.60,0.02}{#1}}
\newcommand{\SpecialCharTok}[1]{\textcolor[rgb]{0.00,0.00,0.00}{#1}}
\newcommand{\StringTok}[1]{\textcolor[rgb]{0.31,0.60,0.02}{#1}}
\newcommand{\VerbatimStringTok}[1]{\textcolor[rgb]{0.31,0.60,0.02}{#1}}
\newcommand{\SpecialStringTok}[1]{\textcolor[rgb]{0.31,0.60,0.02}{#1}}
\newcommand{\ImportTok}[1]{#1}
\newcommand{\CommentTok}[1]{\textcolor[rgb]{0.56,0.35,0.01}{\textit{#1}}}
\newcommand{\DocumentationTok}[1]{\textcolor[rgb]{0.56,0.35,0.01}{\textbf{\textit{#1}}}}
\newcommand{\AnnotationTok}[1]{\textcolor[rgb]{0.56,0.35,0.01}{\textbf{\textit{#1}}}}
\newcommand{\CommentVarTok}[1]{\textcolor[rgb]{0.56,0.35,0.01}{\textbf{\textit{#1}}}}
\newcommand{\OtherTok}[1]{\textcolor[rgb]{0.56,0.35,0.01}{#1}}
\newcommand{\FunctionTok}[1]{\textcolor[rgb]{0.00,0.00,0.00}{#1}}
\newcommand{\VariableTok}[1]{\textcolor[rgb]{0.00,0.00,0.00}{#1}}
\newcommand{\ControlFlowTok}[1]{\textcolor[rgb]{0.13,0.29,0.53}{\textbf{#1}}}
\newcommand{\OperatorTok}[1]{\textcolor[rgb]{0.81,0.36,0.00}{\textbf{#1}}}
\newcommand{\BuiltInTok}[1]{#1}
\newcommand{\ExtensionTok}[1]{#1}
\newcommand{\PreprocessorTok}[1]{\textcolor[rgb]{0.56,0.35,0.01}{\textit{#1}}}
\newcommand{\AttributeTok}[1]{\textcolor[rgb]{0.77,0.63,0.00}{#1}}
\newcommand{\RegionMarkerTok}[1]{#1}
\newcommand{\InformationTok}[1]{\textcolor[rgb]{0.56,0.35,0.01}{\textbf{\textit{#1}}}}
\newcommand{\WarningTok}[1]{\textcolor[rgb]{0.56,0.35,0.01}{\textbf{\textit{#1}}}}
\newcommand{\AlertTok}[1]{\textcolor[rgb]{0.94,0.16,0.16}{#1}}
\newcommand{\ErrorTok}[1]{\textcolor[rgb]{0.64,0.00,0.00}{\textbf{#1}}}
\newcommand{\NormalTok}[1]{#1}
\usepackage{graphicx,grffile}
\makeatletter
\def\maxwidth{\ifdim\Gin@nat@width>\linewidth\linewidth\else\Gin@nat@width\fi}
\def\maxheight{\ifdim\Gin@nat@height>\textheight\textheight\else\Gin@nat@height\fi}
\makeatother
% Scale images if necessary, so that they will not overflow the page
% margins by default, and it is still possible to overwrite the defaults
% using explicit options in \includegraphics[width, height, ...]{}
\setkeys{Gin}{width=\maxwidth,height=\maxheight,keepaspectratio}
\IfFileExists{parskip.sty}{%
\usepackage{parskip}
}{% else
\setlength{\parindent}{0pt}
\setlength{\parskip}{6pt plus 2pt minus 1pt}
}
\setlength{\emergencystretch}{3em}  % prevent overfull lines
\providecommand{\tightlist}{%
  \setlength{\itemsep}{0pt}\setlength{\parskip}{0pt}}
\setcounter{secnumdepth}{0}
% Redefines (sub)paragraphs to behave more like sections
\ifx\paragraph\undefined\else
\let\oldparagraph\paragraph
\renewcommand{\paragraph}[1]{\oldparagraph{#1}\mbox{}}
\fi
\ifx\subparagraph\undefined\else
\let\oldsubparagraph\subparagraph
\renewcommand{\subparagraph}[1]{\oldsubparagraph{#1}\mbox{}}
\fi

%%% Use protect on footnotes to avoid problems with footnotes in titles
\let\rmarkdownfootnote\footnote%
\def\footnote{\protect\rmarkdownfootnote}

%%% Change title format to be more compact
\usepackage{titling}

% Create subtitle command for use in maketitle
\newcommand{\subtitle}[1]{
  \posttitle{
    \begin{center}\large#1\end{center}
    }
}

\setlength{\droptitle}{-2em}

  \title{projeto1}
    \pretitle{\vspace{\droptitle}\centering\huge}
  \posttitle{\par}
    \author{Bernardo}
    \preauthor{\centering\large\emph}
  \postauthor{\par}
      \predate{\centering\large\emph}
  \postdate{\par}
    \date{6 de setembro de 2018}


\begin{document}
\maketitle

Analisar as classes sociais dos cadastrados de acordo com as definições
adotadas pelo IBGE:

Adotando a divisão:

classe alta - A média-alta - B média-intermediária - C média-baixa - D
excluídos(Pobre) - E

\begin{Shaded}
\begin{Highlighting}[]
\KeywordTok{require}\NormalTok{(dplyr)}
\end{Highlighting}
\end{Shaded}

\begin{verbatim}
## Loading required package: dplyr
\end{verbatim}

\begin{verbatim}
## 
## Attaching package: 'dplyr'
\end{verbatim}

\begin{verbatim}
## The following objects are masked from 'package:stats':
## 
##     filter, lag
\end{verbatim}

\begin{verbatim}
## The following objects are masked from 'package:base':
## 
##     intersect, setdiff, setequal, union
\end{verbatim}

\begin{Shaded}
\begin{Highlighting}[]
\KeywordTok{require}\NormalTok{(readr)}
\end{Highlighting}
\end{Shaded}

\begin{verbatim}
## Loading required package: readr
\end{verbatim}

\begin{Shaded}
\begin{Highlighting}[]
\KeywordTok{require}\NormalTok{(ggplot2)}
\end{Highlighting}
\end{Shaded}

\begin{verbatim}
## Loading required package: ggplot2
\end{verbatim}

\begin{Shaded}
\begin{Highlighting}[]
\KeywordTok{require}\NormalTok{(pacman)}
\end{Highlighting}
\end{Shaded}

\begin{verbatim}
## Loading required package: pacman
\end{verbatim}

\begin{Shaded}
\begin{Highlighting}[]
\NormalTok{social <-}\StringTok{ }\KeywordTok{read.csv2}\NormalTok{(}\StringTok{"/home/gudnunes/Área de Trabalho/MD/MD2018/dados/cadastrocivil.csv"}\NormalTok{)}


\NormalTok{  manter=}\KeywordTok{c}\NormalTok{(); }
\ControlFlowTok{for}\NormalTok{ (atr }\ControlFlowTok{in} \KeywordTok{names}\NormalTok{(social)) \{ }
  \ControlFlowTok{if}\NormalTok{ (}\KeywordTok{length}\NormalTok{(}\KeywordTok{table}\NormalTok{(social[,atr]))}\OperatorTok{>}\DecValTok{1}\NormalTok{) \{}
\NormalTok{    manter<-}\KeywordTok{c}\NormalTok{(manter,atr)}
\NormalTok{  \} }
\NormalTok{\}}

\NormalTok{social <-}\StringTok{ }\KeywordTok{select}\NormalTok{(social, manter)}

\NormalTok{manter <-}\StringTok{ }\KeywordTok{c}\NormalTok{( }\StringTok{"Id_SERVIDOR_PORTAL"}\NormalTok{,   }\StringTok{"NOME"}\NormalTok{, }\StringTok{"UF_EXERCICIO"}\NormalTok{,          }
 \StringTok{"DESCRICAO_CARGO"}\NormalTok{, }\StringTok{"CLASSE_CARGO"}\NormalTok{,   }
\StringTok{"REFERENCIA_CARGO"}\NormalTok{, }\StringTok{"NIVEL_CARGO"}\NormalTok{, }
 \StringTok{"SIGLA_FUNCAO"}\NormalTok{, }\StringTok{"NIVEL_FUNCAO"}\NormalTok{, }
 \StringTok{"FUNCAO"}\NormalTok{, }\StringTok{"CODIGO_ATIVIDADE"}\NormalTok{,}
\StringTok{"ATIVIDADE"}\NormalTok{ , }\StringTok{"COD_UORG_LOTACAO"}\NormalTok{,}
 \StringTok{"UORG_LOTACAO"}\NormalTok{, }\StringTok{"ORG_LOTACAO"}\NormalTok{ ,}
\StringTok{"ORGSUP_LOTACAO"}\NormalTok{ ,        }\StringTok{"UORG_EXERCICIO"}\NormalTok{ , }\StringTok{"TIPO_VINCULO"}\NormalTok{ ,         }\StringTok{"JORNADA_DE_TRABALHO"}\NormalTok{ ,}
\StringTok{"DATA_INGRESSO_CARGOFUNCAO"}\NormalTok{, }\StringTok{"DATA_INGRESSO_ORGAO"}\NormalTok{)  }

\NormalTok{social <-}\StringTok{ }\NormalTok{social }\OperatorTok\StringTok{ }\KeywordTok{select}\NormalTok{(manter) }\OperatorTok\StringTok{ }\KeywordTok{filter}\NormalTok{(FUNCAO}\OperatorTok{!=}\StringTok{"Sem informacao"}\NormalTok{)}

\NormalTok{remuneracao <-}\StringTok{ }\KeywordTok{read_csv2}\NormalTok{(}\StringTok{"/home/gudnunes/Área de Trabalho/Mineração de dados/MD2018/dados/Remuneracao.csv"}\NormalTok{)}
\end{Highlighting}
\end{Shaded}

\begin{verbatim}
## Using ',' as decimal and '.' as grouping mark. Use read_delim() for more control.
\end{verbatim}

\begin{verbatim}
## Parsed with column specification:
## cols(
##   .default = col_double(),
##   ANO = col_integer(),
##   MES = col_character(),
##   Id_SERVIDOR_PORTAL = col_integer(),
##   CPF = col_character(),
##   NOME = col_character()
## )
\end{verbatim}

\begin{verbatim}
## See spec(...) for full column specifications.
\end{verbatim}

\begin{verbatim}
## Warning in rbind(names(probs), probs_f): number of columns of result is not
## a multiple of vector length (arg 2)
\end{verbatim}

\begin{verbatim}
## Warning: 1 parsing failure.
## row # A tibble: 1 x 5 col      row col   expected  actual                    file                    expected    <int> <chr> <chr>     <chr>                     <chr>                   actual 1 568621 ANO   an integ~ (*) Verbas indenizatoria~ '/home/gudnunes/Área d~ file # A tibble: 1 x 5
\end{verbatim}

\begin{Shaded}
\begin{Highlighting}[]
\NormalTok{remuneracao <-}\StringTok{ }\NormalTok{remuneracao }\OperatorTok\StringTok{ }\KeywordTok{select}\NormalTok{(Id_SERVIDOR_PORTAL, }\StringTok{`}\DataTypeTok{REMUNERACAO BASICA BRUTA (R$)}\StringTok{`}\NormalTok{, }\StringTok{`}\DataTypeTok{REMUNERACAO APOS DEDUCOES OBRIGATORIAS (R$)}\StringTok{`}\NormalTok{)}
\NormalTok{social<-}\KeywordTok{left_join}\NormalTok{(}\DataTypeTok{x=}\NormalTok{social,}\DataTypeTok{y=}\NormalTok{remuneracao)}
\end{Highlighting}
\end{Shaded}

\begin{verbatim}
## Joining, by = "Id_SERVIDOR_PORTAL"
\end{verbatim}

Aqui podemos ver uma média dos ganhos de todos o cadastrados, depois de
deduções obrigatórias:

\begin{Shaded}
\begin{Highlighting}[]
\NormalTok{social}\OperatorTok{$}\StringTok{`}\DataTypeTok{REMUNERACAO APOS DEDUCOES OBRIGATORIAS (R$)}\StringTok{`}\NormalTok{ <-}\StringTok{ }\KeywordTok{as.double}\NormalTok{(social}\OperatorTok{$}\StringTok{`}\DataTypeTok{REMUNERACAO APOS DEDUCOES OBRIGATORIAS (R$)}\StringTok{`}\NormalTok{)}

\KeywordTok{boxplot}\NormalTok{(social}\OperatorTok{$}\StringTok{`}\DataTypeTok{REMUNERACAO APOS DEDUCOES OBRIGATORIAS (R$)}\StringTok{`}\NormalTok{, }\DataTypeTok{horizontal =}\NormalTok{ T)}
\end{Highlighting}
\end{Shaded}

\includegraphics{apresetançao1_files/figure-latex/unnamed-chunk-2-1.pdf}


\end{document}
